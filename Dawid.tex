\documentclass[a4paper]{article}
\usepackage[left=3.5cm, right=2.5cm, top=2.5cm, bottom=2.5cm]{geometry}
\usepackage[MeX]{polski}
\usepackage[utf8]{inputenc}
\usepackage{graphicx}
\usepackage{color,soul}
\usepackage{enumerate}
\usepackage{amsmath} %pakiet matematyczny
\usepackage{amssymb} %pakiet dodatkowych symboli
\title{Pierwszy dokument LaTeX}
\author{Dawid Nowek}
\date{Październik 2022}
\begin{document}
\maketitle
\newpage

	\section{Zrobiłem tutaj sekcje}
	\paragraph{zrobiłem sobie tutaj paragraf}
	\subparagraph{zrobiłem sobie tutaj subparagraf}
	\subsection{zrobilem sobie subsekcje i zaraz zrobie double sub sekcje}
	\subsubsection{wlasnie zrobilem double subsekcje}
	\subsubsection{teraz druga essa}

\section{teraz druga sekcja}

zaraz zaczne wymienaic rzeczy
\begin{enumerate}
	\item wymienilem pierwsza
	\item teraz druga
		\begin{enumerate}[A]

		\item double item
		\item drugi raz

		\end{enumerate}
\end{enumerate}


\newpage

{\huge\textbf{Cyberpunk 2077}}\newline


Osadzona w otwartym świecie w klimacie science fiction gra RPG oparta na papierowym systemie fabularnym Cyberpunk. Cyberpunk 2077 został opracowany przez studio CD Projekt RED, które wsławiło się kultową serią o Wiedźminie.\newline

Cyberpunk 2077 jest pierwszoosobową grą RPG z otwartym światem, wzbogaconą o elementy FPS-ów. Za jej stworzenie odpowiada polskie studio CD Projekt RED, które zdobyło międzynarodową sławę dzięki bestsellerowemu cyklowi Wiedźmin. Tytuł oparto na licencji Cyberpunka 2020 – gry fabularnej stworzonej w 1990 roku przez Mike’a Pondsmitha.\newline

{\huge\textcolor{red}{\textbf{Fabuła i kreator postaci}}} \newline

Akcja Cyberpunka 2077 toczy się w roku 2077, w fikcyjnej metropolii zwanej Night City położonej na północy kalifornijskiego wybrzeża. Mamy tu do czynienia z dystopijnym retrofuturyzmem rodem z lat 90. XX wieku, w którym zdewastowanym \textbf{światem rządzą brutalne megakorporacje}, a ulice są królestwem gangów i tytułowych cyberpunków – niezależnych buntowników żyjących poza systemem. Ponadto dzięki postępowi technologicznemu ludzie powszechnie modyfikują swoje ciała wszczepami. \newline

\textbf{Wcielamy się w V, czyli początkującego najemnika (lub najemniczkę)} budującego swoją pozycję w surowej rzeczywistości. \textbf{Pewnego razu, wypełniając jedno ze zleceń, bohater(ka) wchodzi w posiadanie czipu skrywającego sekret nieśmiertelności}, który jest obiektem pożądania najpotężniejszych sił cyberpunkowego świata. Protagonistę tworzymy od podstaw, decydując m.in. o płci (co ciekawe, można swobodnie łączyć ze sobą żeńskie i męskie cechy), wyglądzie, czy pochodzeniu postaci (mamy do dyspozycji \textbf{trzy ścieżki życiorysu – nomada, punka lub korpa} – które decydują m.in. o tym, gdzie rozpoczniemy zabawę oraz oferują dodatkowe opcje dialogowe). W kreatorze postaci ustalamy też początkowy poziom podstawowych pięciu atrybutów: budowy ciała, refleksu, zdolności technicznych, inteligencji i opanowania.\newline

{\huge\textcolor{red}{\textbf{Johny Silverhand}}} \newline

Kluczową role w fabule Cyberpunka 2077 odgrywa Johny Silverhand – przebojowy rockerboy (w świecie CP2020 mianem takim określa się rockmana znanego z kontaktów z przestępczym półświatkiem), który zginął w tajemniczych okolicznościach w latach 20. XXI wieku, na wiele lat przed akcją gry. Świadomość tego bohatera została zdigitalizowana i zapisana na czipie, w którego posiadanie wchodzi V. Losy bezkompromisowego gwiazdora są więc związane z protagonistą, ale wcale nie musi to oznaczać, że jest on jego sojusznikiem – może np. próbować przejąć ciało V wbrew jego woli.\newline

{\huge\textcolor{red}{\textbf{Dubbing - obsada}}} \newline


\textbf{W Johny’ego Silverhanda wcielił się Keanu Reeves}, czyli gwiazdor znany m.in. z serii Matrix oraz John Wick. Poza nim głosu użyczyli m.in. Gavin Drei (V - mężczyzna), Cherami Leigh (V – kobieta), Lovensky Jean-Baptiste (Placide), Samuel Barnett (SI Delamaina), Michael-Leon Wooley (Dexter „Dex” DeShawn) czy Jason Hightower (Jackie Welles). \newline \newline
\textbf{W polskiej wersji językowej wystąpili m.in. Michał Żebrowski (Johny Silverhand)}, Kamil Kula (męska wersja V), Lidia Sadowa (żeńska wersja V), Marta Żmuda Trzebiatowska (Judy Alvarez), Przemysław Nikiel (Placide) czy Jacek Król (Jackie Welles). \newline

{\huge\textcolor{red}{\textbf{Gameplay}}}\newline

Cyberpunk 2077 jest grą RPG, w której akcję obserwujemy z perspektywy pierwszoosobowej. Gracz eksploruje rozległy, otwarty świat (możemy chodzić na piechotę lub jeździć samochodami oraz motocyklami) i wykonuje przeróżne zadania, często o nieliniowej strukturze. W grze znalazła się też ciekawa mechanika Braindance, która pozwala odtwarzać oraz badać wspomnienia innych osób. Rozwiązanie to jest często wykorzystywane w różnorakich dochodzeniach.
\newline \newline

 \paragraph{Cyberpunk 2077 {\textbf{Wymagania sprzętowe}}}
 	\subparagraph{{\textbf{Minimalne:}}
 	 Intel Core i5-3570K 3.4 GHz / AMD FX-8310 3.4 GHz 8 GB RAM karta grafiki 4 GB GeForce GTX 970 / Radeon RX 470 lub lepsza 70 GB HDD Windows 7/10 64-bit}






\end{document}